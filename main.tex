%for a more compact document, add the option openany to avoid
%starting all chapters on odd numbered pages
\documentclass[12pt]{cmuthesis}

% This is a template for a CMU thesis.
% The source for this is pulled from a variety of sources and people.
% Here's a partial list of people who may or may have not contributed:
%
%        bnoble   = Brian Noble
%        caruana  = Rich Caruana
%        colohan  = Chris Colohan
%        jab      = Justin Boyan
%        josullvn = Joseph O'Sullivan
%        jrs      = Jonathan Shewchuk
%        kosak    = Corey Kosak
%        mjz      = Matt Zekauskas (mattz@cs)
%        pdinda   = Peter Dinda
%        pfr      = Patrick Riley
%        dkoes    = David Koes

% some useful packages
\usepackage{times}
\usepackage{fullpage}
\usepackage{graphicx}
\usepackage{amsmath}
\usepackage[numbers,sort]{natbib}
\usepackage[hyphens,spaces,obeyspaces]{url}
\usepackage[backref,pageanchor=true,plainpages=false, pdfpagelabels, bookmarks,bookmarksnumbered,
%pdfborder=0 0 0,  %removes outlines around hyper links in online display
]{hyperref}
%\usepackage{subfigure}
\usepackage{braket}

% Approximately 1" margins, more space on binding side
%\usepackage[letterpaper,twoside,vscale=.8,hscale=.75,nomarginpar]{geometry}
%for general printing (not binding)
\usepackage[letterpaper,twoside,vscale=.8,hscale=.75,nomarginpar,hmarginratio=1:1]{geometry}

\setcounter{secnumdepth}{3}
\setcounter{tocdepth}{3}

% Provides a draft mark at the top of the document.
\draftstamp{\today}{DRAFT}

\usepackage{cleveref}
\crefname{figure}{Figure}{Figures}
\Crefname{figure}{Figure}{Figures}
\crefname{section}{Section}{Sections}
\Crefname{section}{Section}{Sections}
\crefname{table}{Table}{Tables}
\Crefname{table}{Table}{Tables}
\crefname{chapter}{Chapter}{Chapters}
\Crefname{chapter}{Chapter}{Chapters}
\usepackage{notoccite}

\usepackage{xcolor}
\hypersetup{
    colorlinks,
    linkcolor={red!50!black},
    citecolor={blue!50!black},
    urlcolor={blue!80!black}
}

\usepackage{tikz}
\usetikzlibrary{patterns}
\usepackage{wrapfig}
\usepackage{tabularx}
\usepackage{booktabs}
\usepackage{siunitx}
\usepackage{subcaption}
\captionsetup[subfigure]{format=hang}
\usepackage{wrapfig}
\usepackage{comment}
%\usepackage{setspace}
%\setstretch{1}


\newcommand{\observeab}{Observe$_{ab}$}
\newcommand{\observec}{Observe$_{c}$}
\newcommand{\orientdecide}{Orient+\allowbreak Decide$_{d}$}
\newcommand{\acte}{Act$_{e}$}
\newcommand{\actfg}{Act$_{fg}$}

\setlength{\parskip}{5pt plus 1pt minus 1pt}%

\newenvironment{captext}{%
   \begin{center}%
     \begin{minipage}{0.95\linewidth}%
       \renewcommand{\baselinestretch}{0.8}%
         \footnotesize\centering}%
   {\renewcommand{\baselinestretch}{1.0}%
      \end{minipage}%
        \end{center}}

\begin {document}
\frontmatter

%initialize page style, so contents come out right (see bot) -mjz
\pagestyle{empty}

\title{ %% {\it \huge Thesis Proposal}\\
{\bf Analyzing the OODA Loop of an Edge-enabled Autonomous Drone System}}
\author{Aditya Chanana}
\date{December 18}
\Year{2024}
\trnumber{}

\committee{
Mahadev Satyanarayanan, Chair \\
Padmanabhan Pillai\\
}

\support{}
\disclaimer{}

% copyright notice generated automatically from Year and author.
% permission added if \permission{} given.

\keywords{edge computing, drones, mobile networks, latency, bandwidth, edge-native applications, computer vision, machine learning}

\maketitle

\begin{dedication}
    Dedicated to my family and girlfriend, for their belief in me.
\end{dedication}

\pagestyle{plain} % for toc, was empty

%% Obviously, it's probably a good idea to break the various sections of your thesis
%% into different files and input them into this file...

\begin{abstract}

    The "Observe, Orient, Decide, Act" (OODA) loop encapsulates the agility of
    cyber-physical or cyber-human systems that depend on continuous iterations
    of these steps. Systems with faster OODA loops react more quickly to
    changes in their environment. This work analyzes the OODA loop of the
    SteelEagle edge-enabled autonomous drone system, identifying bottlenecks
    and opportunities for optimization.
    We also analyze the use of onboard computation with SteelEagle, showing how
    to combine the use of onboard computational abilities, for tasks that
    benefit immensely from a faster OODA loop, and offloading, for tasks that
    are more computationally expensive, leading to an optimal system that
    excels in real-time active vision tasks.

\end{abstract}

\begin{acknowledgments}

    Thank you to my research advisors Dr. Mahadev Satyanarayanan and Dr.
    Padmanabhan Pillai for their guidance and feedback on this project, and
    Thomas Eiszler Jr.  and Mihir Bala for their mentorship.  I would like to
    express my gratitude to my parents, Dr. Rashmi Chanana and Sanjay Chanana,
    who have offered their unwavering love, encouragement, and support at every
    stage of my life.  Thank you to my girlfriend, Isabella Wegner, for your
    belief in me.  I am grateful to my academic advisor Dr. David A. Eckhardt
    for being a reliable source of excellent advice, and his encouragement for
    me to pursue research.

    This material is based upon work supported by the U.S. Army Research Office
    and the U.S. Army Futures Command under Contract No. W519TC-23-C-0003, and
    by the National Science Foundation under grant number CNS-2106862. The
    content of the  information does not necessarily reflect the position or
    the policy of the government and no official endorsement should be
    inferred.

    This work was done in the CMU Living Edge Lab, which is
    supported by Intel, Arm, Vodafone, Deutsche Telekom, CableLabs, Crown
    Castle, InterDigital, Seagate, Microsoft, the VMware University Research
    Fund, IAI, and the Conklin Kistler family fund. Any opinions, findings,
    conclusions or recommendations expressed in this document are those of the
    author and do not necessarily reflect the view of the above funding
    sources.
\end{acknowledgments}


\backrefsetup{disable}
\tableofcontents
\listoffigures

\listoftables
\backrefsetup{enable}

\mainmatter

%% Double space document for easy review:
%\renewcommand{\baselinestretch}{1.66}\normalsize

% The other requirements Catherine has:
%
%  - avoid large margins.  She wants the thesis to use fewer pages,
%    especially if it requires colour printing.
%
%  - The thesis should be formatted for double-sided printing.  This
%    means that all chapters, acknowledgements, table of contents, etc.
%    should start on odd numbered (right facing) pages.
%
%  - You need to use the department standard tech report title page.  I
%    have tried to ensure that the title page here conforms to this
%    standard.
%
%  - Use a nice serif font, such as Times Roman.  Sans serif looks bad.
%
% Other than that, just make it look good...

\chapter{Introduction}
% Goal : 5-10 page talk before the talk
% maybe not 10 pages, but providing a general overview of the paper is the goal here
\section{Motivation}
\label{sec:motivation}
Mobile devices, such as smartphones and smartwatches, have a key limitation---a
limited battery life which allows them to operate for finite periods before
requiring recharging.  Resource-intensive applications, such as those involving
augmented reality or deep learning inference, exacerbate the situation by
accelerating battery depletion.  Consequently, a mobile device cannot match the
sustained performance and resource capacity of stationary computers, which are
free from the requirements of mobility. Mobile devices will always lag behind
static devices in computational ability because of their size and weight
constraints \cite{satya2014}. Increased computational ability, at the same
level of hardware efficiency, demands higher energy consumption. This requires
larger batteries to preserve operating time, thereby increasing weight and
size---which are undesirable for mobile devices.
\Cref{tab:mobility-constraints} shows how mobile devices have lagged behind
servers in computational power over a period of 25 years. This limitation has
been referred to as the ``mobile penalty'', the cost for reduction in
performance required to adhere to mobility constraints
\cite{satya-edge-native}.

\begin{table}[htbp]
    \centering
    \begin{tabular}{@{}lllll}
        \toprule & \multicolumn{2}{c}{\textbf{Typical Server}} & \multicolumn{2}{c}{\textbf{Typical Mobile Device}}\\\midrule
        \textbf{Year} & \textbf{Processor} & \textbf{Speed} & \textbf{Device} & \textbf{Speed}\\\midrule
        1997 & Intel Pentium II & 266 MHz & Palm Pilot & 16 MHz\\
        2002 & Intel Itanium & 1 GHz & Blackberry 5810 & 133 MHz\\
        2007 & Intel Core 2 Quad Q6600 & 2.4 GHz x 4 & Apple iPhone & 412 MHz\\
        2012 & Intel Xeon E5-2690 & 3.8 GHz x 8 & Samsung Galaxy S3 & 1.4 GHz x 4\\
        2017 & Intel Xeon Platinum 8180 & 3.8 GHz x 28 & Google Pixel 2 & 2.35 GHz x 4,\\
             &&&& 1.9 GHz x 4\\
        2022 & AMD EPYC 9654 & 3.7 GHz x 96 & Samsung Galaxy S22 Ultra & 3 GHz x 1,\\
             &&&& 2.5 GHz x 3,\\
             &&&& 1.8 Ghz x 4\\
        \bottomrule
    \end{tabular}
    \begin{captext}
        *While processor efficiency depends on more factors than processor frequency, such as instructions per cycle, it provides a good first-order approximation.
    \end{captext}
    \caption{The Mobility Penalty Over a Period of 25 Years (Adapted from Chen \cite{zchenthesis}, Flinn \cite{flinn2012cyber})}
    \label{tab:mobility-constraints}
\end{table}

Unmanned aerial vehicles (UAVs), or drones, which spend most of their energy on
flight, are particularly affected by this limitation. Larger batteries required
to sustain intensive on-drone computation increases drone weight.  Increased
drone weight leads to an upwards spiral in weight as larger rotors and more
powerful motors are needed to achieve the same amount of lift.  This requires
even larger batteries which, in turn, may require a more reinforced aircraft
structure, and so on. How, then, can we enable mobile devices to do more given
the mobility constraints they are subject to? Advancements in hardware
efficiency provide one possible way forward, but these advancements are slow
and far between. It turns out that there is a way to "cheat" that enables
mobile devices to do more today, leveraging static infrastructure to overcome
the mobility penalty, known as offloading.

A significant body of work in the field of mobile computing has utilized cloud
or cloudlet offloading techniques to address the inherent resource poverty of
mobile devices \cite{satya1996,satya2009}. Cloudlet offload allows mobile
devices to remotely execute computationally expensive tasks on cloudlets that
do not need to be light or small. This allows mobile devices to retain their
low weight and small size, but possess superior computational abilities.

Drones are an excellent category of mobile devices to study because of their
wide range of useful applications as well as their design requirements which
take mobility constraints to the extreme.  We benchmark an autonomous drone
system that leverages offloading techniques to overcome the computational
limitations of consumer-grade drones. We identify its limitations and evaluate
its performance to understand the areas that future work should focus on to
enable more capable drones that can have a wider variety of application.

\section{Overview}
\label{sec:overview}

SteelEagle \cite{bala2024} is an autonomous drone system that leverages
offloading.  Its main tenet is the use of commerical-off-the-shelf (COTS)
drones, which are readily available at low cost but typically have minimal to
no on-board computational resoures.  Utilizing cloudlet offload, SteelEagle
enables COTS drones to perform much more intelligent tasks than what they were
designed for, and exhibit autonomous capabilities typically only found in
larger, more expensive drones.  SteelEagle focuses on active vision tasks which
require the ability to perform live video analytics during flight to determine
the next course of action. SteelEagle drones relay their camera video stream
over a commercial cellular network to a nearby on-ground edge server, a
cloudlet, which runs a computationally expensive pipeline involving neural
networks (\cref{fig:steeleagle-drone-arch}) to perform tasks such as monocular
camera depth perception.  This processing of the raw drone sensor stream allows
obtaining higher level semantic information about the drone's physical
environment. In the case of depth perception, we could infer obstacles that
present the danger of a collision. The drone can then be sent piloting commands
to react, perhaps to actuate to avoid an impending collision.

\begin{figure}[htbp]
\centerline{\includegraphics[width = .6\textwidth]{figs/steeleagle-drone-arch-cropped.pdf}}
\caption{Cloudlet Offload in SteelEagle}
\label{fig:steeleagle-drone-arch}
\end{figure}

An important consideration for SteelEagle is the agility of the resulting
system.  How quickly can a drone respond to a change in its environment? The
end-to-end latency of the entire execution pipeline, including sensing,
offloading, inferencing, decision making, and actuation, defines the agility. A
high end-to-end latency can severely handicap the drone---it must fly at a
higher altitude or at lower speeds to be safe if it takes a long time to
identify obstacles and actuate to avoid them. Drone flight at a higher altitude
precludes close observation, and lower speeds make missions take longer,
limiting the capabilities of the drone. Search and rescue missions in forests,
and missions to aid law enforcement operations in dense cities, for instance,
must fly at low altitudes while avoiding obstacles in the environment that
present a collision risk---trees branches, streetlight poles, utility wires,
and buildings---without impacting mission speed. Unless we can achieve high
agility, SteelEagle drones will struggle to perform these missions well.  Since
these missions are one of the most compelling use cases for autonomous drones,
benchmarking the end-to-end pipeline to determine latency bottlenecks and
identifying opportunities for latency optimization is a very worthy pursuit.

SteelEagle drones currently perform no onboard computation---they are
controlled exclusively over the network.  While this strategy allows treating
consumer-grade non-programmable drones as black boxes, it presents severe
limitations. First, SteelEagle drones struggle in areas with unreliable cell
service and are completely inoperable in regions without cellular coverage.
Second, cloudlet offload imposes an upper bound on drone agility as it adds the
cost of a round-trip latency to a cloudlet. As originally envisioned, cloudlets
are differentiated from clouds because of their network proximity, which allows
application end-to-end response time to be just a few milliseconds. In
practice, the usage of commerical cellular networks for offload to the cloudlet
increases this latency to tens of milliseconds. This limits the agility of the
drone, as the reaction time is at least the round-trip time to the cloudlet.

This thesis performs a comprehensive benchmarking of the SteelEagle system,
revealing the current system bottlenecks and identifying opportunities for
optimization. In recent years, domain-specific system-on-a-chip devices have
become available that provide substantial energy-efficient on-board
computational resources through the inclusion of hardware accelerators in the
chip design. These chips can decode the video stream generated by the drone and
perform analytics using TensorFlow Lite models, and often include 5G and Wi-Fi
connectivity. Using such a chip as a payload on SteelEagle drones allows us to
continue treating the drone as a black box, but employ new cloudlet offload
tactics that result in tighter OODA loops for use cases that can utilize the
hardware accelerators, while retaining the generality offered by cloudlet
offload. We explore the use of onboard computation to mitigate the limitations
of SteelEagle, and explore the resulting impact on drone agility.

\section{Contributions}

The contributions of this thesis include
\begin{itemize}
  \item A discussion of efforts to optimize SteelEagle that yielded a two-fold
      improvement in drone-to-cloudlet latency
  \item A mapping of the SteelEagle pipeline to the OODA loop framework
  \item An analysis of the SteelEagle pipeline's performance from a latency
      and bandwidth perspective
  \item An evaluation of the use of an onboard computation device to improve
      SteelEagle performance and alleviate its limitations
\end{itemize}

\section{Organization}
\Cref{ch:background} provides a detailed background on edge computing and the
SteelEagle autonomous drone system. \Cref{ch:uav-primer} describes the history
of drones and their development over time. We discuss the applications that
drones are used in today, as well as the capabilities of today's drones.  We
also discuss what autonomy means for drones. In
\cref{ch:optimizing-steeleagle}, we describe our experimental setup for
measuring the latency of the SteelEagle system. We include measurements which
provided the insight needed to optimize SteelEagle, allowing for a two-fold
improvement in SteelEagle's drone-to-cloudlet latency. In \cref{ch:voxl}, we
explore the use of an onboard computation device called the Modal AI VOXL to
augment SteelEagle. We conclude with \cref{ch:conclusion}, presenting
concluding remarks and providing a road map for future work.


\begin{comment}
The "Observe, Orient, Decide, Act" (OODA) loop framework devised by military
strategist John R. Boyd provides a framework to structure our investigation
into the end-to-end latency of SteelEagle. According to Boyd, decision-making
happens in a continuous iteration of these steps. Boyd attributed the faster
OODA loop of U.S. pilots flyng F-86s, because of the bubble-shaped canopy
offering better visibility and hydraulic controls that allowed for easier
switching between manoeuvres, as the reason the slower F-86s fared better than
the North Korean MiG-15s during the Korean War \cite{morton1995}. A system with
a tighter OODA loop corresponds to a more agile drone. Instead of measuring
just the overall system latency, performing a break down of the latency across
the OODA steps provides more insight into system latency bottlenecks.
\end{comment}

\chapter{Background and Related Work}
\label{ch:background}

This chapter presents background information on the field of edge computing
(\cref{sec:bg-edge}), the SteelEagle autonomous drone system
(\cref{sec:steeleagle-bg}), and the OODA loop framework (\cref{sec:ooda-loop}).

\section{Edge Computing}
\label{sec:bg-edge}

There has been a huge increase in the number of mobile and Internet of Things
(IoT) devices in recent years, driven by advancements in sensor, networking,
and processing technologies.  These innovations have made devices smaller, more
affordable, and more versatile, enabling their integration into nearly every
aspect of modern life. As explained in \cref{sec:overview}, despite the
innovations, these mobile devices remain resource-poor relative to static
resources. A wearable device like a smartwatch, for example, has limited
battery life, which constraints its ability to sustain prolonged conversation
with an onboard AI virtual assistant. In 1997, Noble et al extended an adaptive
application-aware framework called Odyssey, which provides remote data access
to mobile clients, to perform speech recognition on a resource-constrained
mobile device by offloading compute to a remote server \cite{noble1997}. This
offloading technique allows a mobile device to circumvent its resource
limitations.

But a key question remains---where to offload? The cloud is an
appealing choice today because of its on-demand and scalable nature. However,
cloud computing resources are consolidated into datacenters at a small number
of geographic locations to leverage economies of scale, increasing the distance
between ends users and the cloud and thus increasing network latency. This can
be a dealbreaker for latency-sensitive interactive applications such as
augmented reality. An emperical study conducted with 2,504 Amazon EC2 clients
found that more than 60\% of the clients experienced latencies higher than 40
ms \cite{choy2012}. However, immersive augmented reality on a wearable device
has a latency bound of 16 milliseconds \cite{ellis2004}.
To attain crisp interactive application response,
Satyanarayanan et al introduced the novel paradigm of edge computing
\cite{satya2009}.


Edge computing is a paradigm in which substantial computing and storage
resources---referred to as "cloudlets"---are situated at the edge of the
Internet in close proximity to mobile devices, sensors, end users, and Internet
of Things (IoT) devices. This proximity offers several key advantages \cite{satya2017}:
\begin{itemize}
    \item It allows offloading to truly shine, enabled by the low latency, high
bandwidth, and low jitter links to the cloudlets
    \item It reduces the bandwidth demand that IoT devices like video cameras place on the cloud, thus increasing scalability \cite{premsankar2018}
    \item A cloudlet can enforce a user's privacy policies specified for their IoT sensor data before the extracted information and metadata obtained from it is sent to the cloud
    \item Cloudlets help mitigate cloud outages by serving as a fallback service
\end{itemize}

\section{SteelEagle Autonomous Drone System}
\label{sec:steeleagle-bg}

\begin{figure}[htbp]
\centering
\begin{subfigure}[t]{0.3\textwidth}
\centering
\includegraphics[height=1.6in]{sec2023-figs/fig-omega-internals.jpg}\\
\caption{The Samsung Galaxy Watch and the Onion Omega 2 LTE}
\end{subfigure}
\hspace{3em}
\begin{subfigure}[t]{0.3\textwidth}
\centering
\includegraphics[height=1.6in]{sec2023-figs/fig-omega-harness.jpg}
\caption{Onion Omega 2 LTE mounted on Parrot Anafi}
\end{subfigure}
\caption{The Hardware Used in SteelEagle (from Bala et al \cite{bala2024})}
\label{fig:galaxy_watch}
\end{figure}

Autonomous drones perform tasks such as navigating between waypoints and
tracking moving objects without the need for a human pilot.  However,
autonomous drones today are large and expensive. Lightweight drones are more
appealing, as they present a smaller public safety hazard and thus face fewer
regulations. The FAA, for instance, has pre-authorized flights over people and
vehicles by drones weighing less than 250 g. On the other hand, cheaper drones
will help accelerate the uptake of autonomous drones in scenarios that stand to
benefit the most from their abilities. Search and rescue operations, as well as
wildlife conservation efforts that involve monitoring wildlife populations,
will benefit immensely from the abilities of drones to cover large areas
quickly. The potential for good increases exponentially with a swarm of drones
working cooperatively \cite{scherer2015}. In rural areas with limited law
enforcement resources, for instance, a swarm of drones could quickly scan
large swathes of land looking for an abducted child.

Over time, autonomous drones will
become cheaper and lighter as new ASIC designs are developed that are more
energy efficient and mass production lowers costs. Until then, leveraging edge
computing to add autonomous features to lighter consumer-grade drones at a much
lower price tag is a very appealing proposition. This is the software-hardware
co-evolution path that Satyanarayanan et al outlined, presenting offloading as
a way to "cheat" until ASIC designs are available \cite{satya21}.

Bala et al \cite{bala2024} have demonstrated that even drones with a monocular
camera can perform tasks such as object tracking and depth inference
reasonablly well. In their setup, Bala et al initially used a Samsung Galaxy
smartwatch as a communications relay, mounted on top of a Parrot Anafi drone,
for a total takeoff weight of about 360 g. The Samsung Galaxy smartwatch is
appealing because it is an enclosed system, including a battery and an
enclosure protected from the elements. It also has the ability to run Android
applications onboard.  However, the constant LTE tranmission on the watch
caused it to hit its thermal limits, set so that the watch can be safely worn
on the human wrist, and shut down. \Cref{fig:watch-themal-curve} shows the
increase in temperature of the watch for a frame rate of 0.7 and 2.

As a result, the Samsung Galaxy smartwatch
could sustain a very low frame rate.

\begin{figure}[htbp]
\centering
\includegraphics[width=0.6\textwidth]{sec2023-figs/fig-thermal-curve.png}
\caption{The Samsung Galaxy Payload Thermal Curve (from Bala et al \cite{bala2024})}
\label{fig:watch-themal-curve}
\end{figure}

As an alternative, Bala et al used the Onion Omega 2 system-on-a-chip device
\cite{onionomega2}. While the Omega 2 does not have the thermal limitations
present in the Galaxy smartwatch, it has very weak computational capabilities.
Intended for use as an IoT module, the Omega 2 runs the 580MHz MIPS 24KEc CPU
and has only 16 MB of flash storage. In the SteelEagle setup, a VPN tunnel
between the cloudlet and the Omega 2 over 4G LTE cellular allows communication
with the drone that is connected to the Omega 2 over Wi-Fi. The Omega 2's role
is to route packets between its Wi-Fi and LTE network interfaces, truly acting
as just a communications relay.

\section{OODA Loop Framework}
\label{sec:ooda-loop}

\begin{figure}[htbp]
\centering
\includegraphics[width=0.4\textwidth]{figs/ooda-loop.png}
\caption{The OODA Loop}
\label{fig:ooda-loop}
\end{figure}

The "Observe, Orient, Decide, Act" (OODA) loop framework devised by military
strategist John R. Boyd provides a framework to structure our investigation
into the end-to-end latency of SteelEagle. It consists of four stages:

\begin{itemize}
    \item \textbf{Observe.} Involves obtaining new information about the environment
    \item \textbf{Orient.} Analysis and interpretation of the obtained information
    \item \textbf{Decide.} Choosing a course of action
    \item \textbf{Act.} Execution of the chosen action
\end{itemize}

According to Boyd, decision-making happens in a continuous iteration of these
steps. Boyd attributed the faster OODA loop of U.S. pilots flyng F-86s, because
of the bubble-shaped canopy offering better visibility and hydraulic controls
that allowed for easier switching between manoeuvres, as the reason the slower
F-86s fared better than the North Korean MiG-15s during the Korean War
\cite{morton1995}. In the context of autonomous drones, a drone with a tighter
OODA loop corresponds to a more agile drone. It means that the drone is able
to react faster to changes in its environment.

Instead of measuring just the overall system latency, performing a
break down of the latency across the OODA steps provides more insight into
system latency bottlenecks.

\begin{comment}
\section{Background: SteelEagle Benchmarks}
Bala et al performed experiments to measure the agility of the SteelEagle
system. The experiment setup involves setting up a stationary drone in a lab
setting, with its camera pointed at a display connected to the cloudlet, showing
the current timestamp in milliseconds. The drone camera captures images of this
timestamp and transmits them to the cloudlet through the SteelEagle pipeline.
\end{comment}

\chapter{Primer on UAVs}

\section{History of Drones}
\label{sec:drone-history}

Unmanned aerial vehicles (UAVs), or drones, are aircraft that can be operated
remotely without the need for a human pilot on board. While there were many
early pilotless aircraft, the first remote controlled aircraft appeared during
the First World War, developed by Britain and the US in 1917
\cite{dronehistory}. Many of these early drones were used as anti-aircraft
gunnery training targets. In the 1930s, the term "drone" arose inspired by the
de Havilland DH82B Queen Bee (\cref{fig:queen-bee}), designed as a low-cost
radio-controlled target aircraft, which saw over four hundred units built by
1943. The Queen Bee was the first drone designed with the ability to return to
ground safely and be reused \cite{queenbee, pbs05}.

\begin{figure}[htbp]
\centerline{\includegraphics[width = .6\textwidth]{figs/queenbee.png}}
\caption{de Havilland DH82B Queen Bee \cite{baesystems}, the first radio-controlled drone}
\label{fig:queen-bee}
\end{figure}

These early drones were fixed-wing aircraft that were used primarily for combat
or training. From the 1970s, drones such as the Ryan Model 147 Lightning
Bugdrones were developed for use in surveillance missions. These drones carried
a camera and could fly for hours at high altitude \cite{pbs12}. The Pioneer
UAV, introduced in 1986, saw extensive use in the Gulf War. Today, the General
Atomics MQ-1 Predator can fly for over 14 hours performing surveillance
missions using an array of sensors, including infrared cameras.

The turn of the century saw the use of small drones in civillian settings, with
technology advances making them cheaper to produce. Drones, particularly
quadcopters (\cref{fig:dji-mini-4k}), started being used for mapping, aerial
photography, industrial inspections and security, and precision agriculture
\cite{giones2017}. Compared to previous fixed-wing military drones, quadcopters
offer superior maneuverability and the ability to hover. They utilize four
motor-rotors, with two spinning clockwise and the other two counterclockwise.
Variations in motor speed allow for precise hovering and maneuverability. This
makes quadcopter drones well-suited for both indoor and outdoor use.  Since the
2010s, drones have evolved significantly, with a range of consumer photography
and racing drones easily available to consumers, at a wide range of price
points. Drones are commonly used for filming and recreation, and have also seen
use in logistics, to deliver packages. The scale of their adoption is
immense---The Federal Aviation Administration (FAA) has received registrations
for almost 800,000 drones \cite{faa_drones_2024}.  This figure excludes
thousands of hobbyist drones weighing under 250 g that do not require FAA
registration.

\section{The capabilities of today's drones}
\label{sec:drone-capabilities}

\begin{figure}[htbp]
\centerline{\includegraphics[width = .5\textwidth]{figs/dji-mini-4k.jpg}}
\caption{DJI Mini 4K}
\label{fig:dji-mini-4k}
\end{figure}

Today, the set of features available in consumer-grade drones is impressive.
The DJI Mini 4K, for instance, is priced at \$299, weighs under 250g, records
video at 4K 30 FPS, and has a maximum flight time of about 30 minutes
\cite{dji_mini_4k}. Most drone's today have the ability to fly
semi-autonomously in addition to the ability to be controlled remotely by a
human. As shown in \cref{fig:drone-components}, drones have advanced
microprocessors that abstract away the lower-level mechanics of quadcopter
drone flight. Components such as the Electronic Speed Controller (ESC) enable
precise control of the drone's brushless electric motors. On-board positioning
systems such as Global Navigation Satellite System (GNSS) receivers give the
drone the ability to navigate between waypoints.  Using inputs from on-board
intertial measurement unit (IMU) sensors such as gyroscopes, accelerometers,
and magnetometers, drones can determine their orientation, acceleration, and
rotation, allowing them to adjust their motors in real-time to counter wind and
hover in a fixed position. They can also typically takeoff and land
autonomously. On-board flight control software such as PX4 or ArduPilot, also
known as an "autopilot," makes these higher-level functions possible, taking
inputs from the IMU and GNSS sensors and outputting control commands to the
ESC.

\begin{figure}[htbp]
\centerline{\includegraphics[width = .6\textwidth]{figs/drone-components-crop.pdf}}
\caption{Components of a drone \cite{giones2017}}
\label{fig:drone-components}
\end{figure}

Drones are equipped with a variety of cameras, such as monocular and stereo.
Monocular cameras capture images and videos from a single point of view, and
are common in consumer aerial photography drones. Monocular cameras are often
mounted on a gimbal, which stabilizes the drone's camera during motion and also
allows the ability to capture different viewing angles through gimbal motion.
Stereoscopic cameras, on the other hand, offer multiple point of views. Having
more than one camera allows the use of epipolar geometry and triangulation to
determine distance from objects efficiently, and perform obstacle avoidance.
The DJI Mini 4 Pro, for instance, has multiple binocular cameras, facing
forward, backward, sideways, upwards, and downwards. This allows the Mini 4 Pro
to perform omnidirectional obstacle sensing. During manual pilot flight, the
drone offers pilot assistance features that stop the drone if it is headed for
an imminent collision, and also provide the option to circumvent obstacles
automatically altogether. Drones equipped with monocular cameras typically lack
these advanced obstacle avoidance systems, but are cheaper and more common.

While consumer-grade drones offer many semi-autonomous features, complex
fully-autonomous features are limited to more expensive commerical drones.  For
example, a consumer drone could be instructed to navigate to a given GPS
coordinate or, in some advanced drones, even track an on-ground target. But
complex missions, such as patrolling a set of waypoints while searching for an
on-ground target, and transitioning to track the target once it is detected,
remain out of the reach of consumer drones.

Consumer drones are typically controlled over Wi-Fi using a controller or a
smartphone app, and typically lack cellular connectivity, limiting their
flying range.



\chapter{Optimizing SteelEagle Performance}

As explained in \cref{sec:overview}, the performance of the SteelEagle system
determines its versatility. A drone that is slow to react will be unable to
effectively navigate obstacle-dense environments. The chances of the drone
losing track of a fast on-ground target that is moving erratically increases
substantially if the drone is slow in keeping the target centered in its view.
Given the importance of performance, this chapter details how performance is
defined for SteelEagle, how it is measured, and work done to identify
performance bottlenecks and optimize the system.

\begin{figure}[htbp]
\centerline{\includegraphics[width = .5\textwidth]{figs/fig-simplified-arch.png}}
\caption{SteelEagle Edge Offload Pipeline}
\label{fig:simplified-arch}
\end{figure}

\section{How is performance defined in SteelEagle?}
\label{sec:steeleagle-performance-def}

The performance of SteelEagle is determined by the end-to-end latency and
throughput of the system. \cref{sec:overview} explained how drones in
SteelEagle are treated as thin clients, with the use of a communications relay
to make up for the lack of cellular connectivity on commerical-off-the-shelf
(COTS) drones. The sensor stream from the drone is forwarded to the
communications relay over Wi-Fi, which in turn forwards it to the cloudlet over
4G LTE. After performing inference on the received data, the cloudlet sends
back piloting commands to the drone, via a hop through the communications relay
(\cref{fig:simplified-arch}). As a result, the end-to-end performance is
determined by several components of the pipeline:
\begin{enumerate}
    \item[(a)] On-drone sensing
    \item[(b)] On-drone pre-processing
    \item[(c)] Tranmission to cloudlet
    \item[(d)] Cloudlet processing
    \item[(e)] Transmission to drone
    \item[(f)] Drone post-processing
    \item[(g)] Drone actuation
\end{enumerate}

Each of these components has a latency associated with it, and a throughput
that it is capable of. The performance of components (a), (b), (f), and (g) is
fixed for a given drone. Factors such as the drone camera's sensor readout time
and shutter speed used affect the frame capture latency, and thus determine the
performance of component (a). Component (b) consists of any processing of the
sensor data before it is transmitted from the drone. The generation of an H.264
stream, for instance, adds latency since the compression process is
computationally intensive, involving analysis of deltas between successive
frames. For components (c) and (e), performance is determined by the cellular
network used for communication between the relay and the cloudlet. While there
is variance associated with the performance of these components, factors such
as the number of network hops, distance between the relay to the cloudlet, and
the network signal strength available to the relay largely determine the mean
value of their latency and throughput over a longer period of time. The
performance of component (d) can vary largely based on whether decoding of the
drone stream data is required and type of inferencing that is performed. A DNN
with a complex architecture, for instance, can take much longer for inference
than a traditional computer vision approach. The same inference can also take
less time on a more powerful edge server.

\begin{figure}[htbp]
\definecolor{observe-color}{RGB}{175,208,149}
\definecolor{orient-color}{RGB}{255, 255, 166}
\definecolor{decide-color}{RGB}{255,170,149}
\definecolor{act-color}{RGB}{224,194,205}
\centering
\includegraphics[width = .6\textwidth]{figs/fig-ooda-loop.pdf}\\
\begin{tikzpicture}
    \draw[fill=observe-color] (1.0,0) rectangle (1.3,0.3);
    \node[right] at (1.3,0.15) {\small Observe};

    \draw[fill=decide-color] (2.9,0) rectangle (3.2,0.3);
    \node[right] at (3.2,0.15) {\small Orient \& Decide};

    \draw[fill=act-color] (6.1,0) rectangle (6.4,0.3);
    \node[right] at (6.4,0.15) {\small Act};
\end{tikzpicture}
\caption{Mapping the SteelEagle pipeline to the OODA loop}
\label{fig:ooda-mapping}
\end{figure}

\Cref{fig:ooda-mapping} shows how these components can be mapped to the various
stage of the OODA pipeline. The ``Observe'' phase includes components (a), (b),
and (c). Components (d) maps to the ``Orient'' and ``Decide'' phases. Finally,
components (e), (f), and (g) correspond to the ``Act'' phase.

\begin{figure}[htbp]
\centerline{\includegraphics[width = .5\textwidth]{figs/fig-ooda-nomenclature.pdf}}
\caption{Measurable Components of the OODA Loop}
\label{fig:ooda-nomenclature}
\end{figure}

We are limited in our ability to individually measure the contribution of
components of the OODA loop. The drone sensing and pre-processing, components
(a) and (b), for instance, cannot be measured individually because the drone
runs closed source software that does not allow the ability to insert software
instrumentation. Similarly, drone post-processing and actuation must be
measured together.  \Cref{fig:ooda-nomenclature} shows the parts of the
pipeline that can be measured in red.

\section{Measuring the performance of the SteelEagle pipeline}
\label{sec:steeleagle-performance-measurement}

\begin{figure}[htbp]
\centerline{\includegraphics[width = .4\textwidth]{figs/bala_latency.pdf}}
\caption{Original SteelEagle Latency \cite{bala2024}}
\label{fig:steeleagle-original-latency}
\end{figure}

\begin{wrapfigure}{r}{0.25\textwidth}
    \centering
    \includegraphics[width=0.2\textwidth]{figs/parrot-anafi.pdf}
    \caption{Parrot Anafi}
    \label{fig:parrot-anafi}
\end{wrapfigure}
The starting point of our investigation into the performance of the SteelEagle
system is the mean drone-to-cloudlet latency of 933 ms reported by Bala et al
in their benchmarking of the SteelEagle system
(\cref{fig:steeleagle-original-latency}). The latency was obtained using the
Parrot Anafi drone (\cref{fig:parrot-anafi}), which transmits a 720p H.264 RTSP
stream at 30 FPS over UDP from its monocular camera.  The Anafi uses a slice
encoding and intra-refresh scheme that disperses keyframe slices across
multiple network packets \cite{anafi_white_paper}. To reduce network bandwidth
requirements, it generates an H.264 compressed video stream using onboard
hardware from the raw frames that it obtains from its camera. Consequently,
decoding of this H.264 stream is required to obtain individual video frames.
The setup uses the Onion Omega 2 as the communications relay, since the Parrot
Anafi lacks cellular connectivity.

\begin{wrapfigure}{r}{0.25\textwidth}
    \centering
    \vspace{-1mm}
    \includegraphics[width=0.2\textwidth]{figs/onion-omega.png}
    \caption{Onion Omega}
    \label{fig:onion-omega}
    \vspace{-1mm}
\end{wrapfigure}
Calculating drone-to-cloudlet latency is challenging since the Anafi, being a
commerical product that restricts modification of its onboard software,
functions as a black box. The inability to add software instrumentation on the
drone makes it difficult to determine when a given frame was transmitted from
the drone. To circumvent this restriction, Bala et al adapted the technique
George et al used for measuring motion-to-photon latency in augmented reality
\cite{george20}.


\begin{figure}[htbp]
\centerline{\includegraphics[width = .8\textwidth]{figs/mtp_pipeline.png}}
\caption{Technique for measuring end-to-end latency}
\label{fig:latency-measuring-technique}
\end{figure}
As shown in \cref{fig:latency-measuring-technique}, the drone is kept
stationary in a lab setting with its camera pointed at a display connected to
the cloudlet showing the current Unix timestamp at millisecond granularity. The
drone captures images containing the timestamp displayed and transmits them to
the cloudlet through the SteelEagle pipeline. The cloudlet records the
timestamp at which it receives each frame, storing the frame along with this
timestamp. To obtain the drone-to-cloudlet latency, these saved frames are
post-processed to compute the difference between the timestamps shown in each
frame and the timestamps at which they were received.

The mean latency of 933 ms reported by Bala et al is incredibly high.

\Cref{tab:latency_summary}

\begin{table}[htbp]
    \centering
    \caption{Drone-to-Cloudlet SteelEagle Latency (in ms)}
    \label{tab:latency_summary}
        \sisetup{
        detect-all,
        table-number-alignment = center,
        input-decimal-markers = {.},
        group-separator={,},
        group-minimum-digits=4
    }
    \begin{tabularx}{0.6\linewidth}{@{}
        X
        c%S[table-format=3.0]
        S[table-format=3.0]
        S[table-format=3.0]
        S[table-format=3.0]
        S[table-format=4.0]@{}
    }
    \toprule
    \textbf{Configuration} & \textbf{Average} & \textbf{Median} & \textbf{p95} & \textbf{Min} & \textbf{Max} \\
    \midrule
    \multicolumn{6}{@{}l}{\textbf{Cloudlet}} \\
    FFmpeg & 888 {\small $\pm$ 30} & 887 & 917 & 838 & 1030 \\
    PDrAW   & 380 {\small $\pm$ 15} & 379 & 402 & 344 & 420 \\
    %WiFi + FFmpeg & 841.52 & 841 & 868.3 & 16.79 & 807 & 879 \\
    %WiFi + PDrAW & 340.82 & 338.5 & 373.5 & 21.82 & 298 & 419 \\
    \midrule
    \multicolumn{6}{@{}l}{\textbf{AWS Small}} \\
    FFmpeg & 536 {\small $\pm$ 75} & 521 & 600 & 473 & 936 \\
    PDrAW   & 429 {\small $\pm$ 21} & 427 & 467 & 387 & 475 \\
    \midrule
    \multicolumn{6}{@{}l}{\textbf{AWS Big}} \\
    FFmpeg & 870 {\small $\pm$ 20} & 868 & 900 & 837 & 925 \\
    PDrAW   & 367 {\small $\pm$ 20} & 365 & 392 & 327 & 416 \\
    \bottomrule
    \end{tabularx}\\
    \vspace{0.1in}
    \footnotesize
    50 samples obtained for each configuration.
\end{table}


\chapter{Utilizing Onboard Compute In SteelEagle}

This chapter discusses work to replace the communications relay used in
SteelEagle to the Modal AI VOXL 2. The VOXL 2 offers significant computational
capability, allowing the execution of machine learning models.  We start by
characterizing this shift in offloading strategy in more general terms.

\section{The Computation Offload Spectrum}
\begin{figure}[htbp]
\centerline{\includegraphics[width = .8\textwidth]{figs/offload-spectrum-crop.pdf}}
\caption{Devices Placed in the Computation Offloading Spectrum}
\label{fig:offload-spectrum}
\end{figure}
Devices today leverage offloading to various degrees.
\Cref{fig:offload-spectrum} shows how devices can be placed in an offloading
spectrum, with thin clients found towards the left of the spectrum and thick
clients to the right. A thin client, as opposed to a thick client, has minimal
compute and can be made smaller, lighter, and cheaper. Virtual desktop
infrastructure (VDI), for example, leverages pure offload by providing remote
access to desktops hosted on centralized servers.  This reduces the
computational demands on the client device, enabling the use of
computationally-intensive applications on weaker and older ``thin'' hardware,
reducing costs. Similarly, internet-of-things (IoT) sensors, such as video
cameras, cirumvent their limited storage and computational ability by
leveraging pure offload, transmitting their sensor streams to the cloud for
storage and analysis.

On the other end of the spectrum, we find devices such as gaming consoles,
desktop computers, and autonomous cars. Gaming consoles and desktop computers
are expected to have crisp interaction, which is not achievable by offloading
to cloud. Offloading to cloudlets, which can provide low latency, is not a
feasible option for these devices currently because of the absence of
widespread cloudlet infrastructure.  Gaming consoles and desktop computers can
provide high-levels of onboard computational power because they are not subject
to mobility constraints.  Although autonomous cars are mobile, they typically
have sufficient energy available to perform intensive onboard computations
since the majority of energy is needed to propel the car forward. They must
also perform real-time decision making reliably even in the absence of network
connectivity, which makes offloading less appealing.

In the middle of the spectrum, we find devices that perform partial offloading.
These devices have sufficient computational power to function when offloading
is unavailable or not advantageous, but they also offload computation in other
cases. Smartphone AI voice assistants such as Apple's Siri, for example,
originally offloaded all computation to the cloud. Over time, it evolved to
utilize on-device capabilities to perform simpler tasks, allowing its use in
the absence of an internet connection.  Chromebooks were designed with a focus
on the use of cloud-based services, allowing them to be fitted with lower
processing power, memory, and storage capacity compared to traditional laptops.

\subsection{How should applications decide the offloading strategy?}
\label{sec:deciding-offloading-strategy}

Choosing an offload strategy for a given application can be done based on some
key attributes that are considered important for the application:
\begin{enumerate}
    \item \textbf{Mobility requirements}: is the application subject to
        mobility constraints?
    \item \textbf{Disconnected operation}: does the application need to work reliably
        in the face of network disconnection or degradation? Is degraded operation
        acceptable during disconnected operation?
    \item \textbf{Latency requirements}: is the application latency-sensitive?
        Are there tasks that are computationally intensive but can tolerate
        high latency if a nearby cloudlet is unavailable?
    \item \textbf{Bandwidth constraints}: how much network bandwidth is
        available to the application?
    \item \textbf{Cost requirements}: is low cost for devices a priority in
        this application?
\end{enumerate}

These attributes encapsulate the benefits and limitations of offloading.
Offloading is typically only relevant for mobile devices. A mobile device that
is dependent entirely on offloading will be unable to function adequately when
offloading is impossible altogether because of network disconnection, or
impacted because of degraded network conditions, such as high latency and low
bandwidth availability. Applications that are mission-critical would be wise to
equip devices with sufficient computational resources to support an acceptable
level of function when adverse network conditions impact offloading.
Applications that are not mission-critical can rely on pure offload to benefit
from the cost savings resulting from using devices that are not required to
have significant computational resources.

Offloading allows mobile devices to achieve a much higher level of
functionality that their onboard hardware allows for. Equipping mobile devices
with the hardware needed to maintain the same level of functionality during
offload degradations reduces the benefit of offloading---the mobile device is
weighed down with the additional size and weight of more complex hardware that
remains unused when offloading is possible.

If offload degradations are infrequent, a more reasonable approach is to
utilize the inferior on-device computational resources to compute results that
are inferior according to an output-specific metric, compared to those obtained
using offloading. For predictions from a machine learning model, the metric
could be the accuracy of the prediction. In previous work, the concept of
\textit{fidelity} has been used in this context, defined as the degree to which
the results differ according to the metric \cite{noble1997}.

\subsection{The dynamic nature of partial offload}

Network conditions affect the performance of offloading, but they are dynamic
in nature. Mobile devices utilizing partial offload, then, must decide an
offloading strategy at runtime. But the decision is not binary---exclusively
using onboard compute or remote compute resources.  As
\cref{fig:offload-spectrum} shows, there is a wide range between the two ends
of the spectrum. Depending on the current network conditions, partial-offload
mobile devices can utilize a combination of onboard and remote computational
resources that results in optimal performance.

Mobile devices can achieve this through an optimal partitioning of applications
into local and remote components. Flinn et al tackled the issue of developing
applications that can support devices with different capabilities and dynamic
execution conditions with a self-tuning remote execution system. The system,
called \textit{Spectra}, continuously monitors the application's resource
supply and demand and suggests which components of an application should be
executed remotely, taking into account factors such as performance, energy
usage, and fidelity \cite{flinn2002}. This work requires the application
developer to partition the application, specifying application components
that could benefit from remote execution.

\subsection{Offloading shaping}
\label{sec:offload-shaping}

Hu et al \cite{hu2015} showed that there is value to doing additional onboard
computation when offloading, which was not originally part of the application.
The approach, called \textit{offload shaping}, attempts to conserve wireless
bandwidth and energy through the process of \textit{early discard}. The process
involves selectively sending inputs for offload computation based on an
application-specific metric of value assigned to each input.  In the case of
video analytics, for instance, onboard computation can detect blurry frames and
not transmit them to the cloudlet. Hu et al showed that object recognition does
not perform well on blurry frames, and so it is a waste of bandwidth and energy
to transmit these blurry frames to a cloudlet for computation.

\section{A Partial Offload Pipeline for SteelEagle}

SteelEagle drones can currently be placed in the ``pure offload'' part of the
spectrum in \cref{fig:offload-spectrum}, since they do not perform any on-board
computation other than the encoding of the raw camera feed to an RTSP H.264
video stream that is transmitted to a cloudlet. As discussed in
\cref{sec:deciding-offloading-strategy}, a pure offload strategy does not
support disconnected operation, and impacts the application considerably as
network conditions degrade.

\begin{figure}[htbp]
\centerline{\includegraphics[width = .5\textwidth]{figs/voxl2.png}}
\caption{Modal AI VOXL 2}
\label{fig:voxl2}
\end{figure}

Replacing the Onion Omega 2 with the Modal AI VOXL 2 (\cref{fig:voxl2}) as the
communications relay in the SteelEagle pipeline will allow us to pursue a
partial offloading strategy that allows SteelEagle drones to gracefully react
to changes in network conditions. Intended for use as an AI autopilot in custom
drones, the VOXL 2 features a Qualcomm QRB5165 processor and a PX4 flight
controller. We do not utilize the PX4 flight controller on the VOXL.
\Cref{tab:voxl2-specs} provides information about the computational resources
available on the VOXL 2.

\begin{table}[htbp]
    \centering
    \begin{tabular}{@{}ll@{}}
        \toprule
        \textbf{} & \textbf{ModalAI VOXL 2}\\
        \midrule
        \textbf{Architecture} & 64-bit ARM\\
        \textbf{CPU} & Qualcomm Kyro 585 (7nm, released Dec 2019)\\
                     & 1 x 2.84 GHz high-performance core\\
                     & 3 x 2.42 GHz high performance cores\\
                     & 4 x 1.80 GHz low-performance cores\\
        \textbf{ISP} & Qualcomm Spectra 480\\
        \textbf{GPU} & Qualcomm Adreno 650, with support for OpenCL\\
        \textbf{DSP} & Qualcomm Hexagon 698\\
        \textbf{Memory} & 8 GB\\
        \textbf{Weight} & 16 grams\\
        \textbf{Power consumption} & 4-5 W under high load\\
        \textbf{Operating system} & Ubuntu 18.04\\
        \bottomrule
    \end{tabular}
    \caption{Technical specifications of the Modal AI VOXL 2 system-on-a-chip}
    \label{tab:voxl2-specs}
\end{table}

\begin{figure}[htbp]
\definecolor{observe-color}{RGB}{175,208,149}
\definecolor{orient-color}{RGB}{255, 255, 166}
\definecolor{decide-color}{RGB}{255,170,149}
\definecolor{act-color}{RGB}{224,194,205}
\centering
\includegraphics[width = .7\textwidth]{figs/fig-voxl-ooda-loop-crop.pdf}
\begin{comment}
\begin{tikzpicture}
    \draw[fill=observe-color] (1.0,0) rectangle (1.3,0.3);
    \node[right] at (1.3,0.15) {\small Observe};

    \draw[fill=decide-color] (2.9,0) rectangle (3.2,0.3);
    \node[right] at (3.2,0.15) {\small Orient \& Decide};

    \draw[fill=act-color] (6.1,0) rectangle (6.4,0.3);
    \node[right] at (6.4,0.15) {\small Act};
\end{tikzpicture}
\end{comment}
\begin{tikzpicture}
    % Observe box
    \draw[fill=observe-color] (1.0,0) rectangle (1.3,0.3);
    \node[right] at (1.3,0.15) {\small Observe};

    % Orient box with yellow hatching
    \draw[fill=white, pattern=north east lines, pattern color=yellow]
        (2.9,0) rectangle (3.2,0.3);
    \node[right] at (3.2,0.15) {\small Orient};

    % Decide box
    \draw[fill=decide-color] (4.5,0) rectangle (4.8,0.3);
    \node[right] at (4.8,0.15) {\small Decide};

    % Act box
    \draw[fill=act-color] (6.2,0) rectangle (6.5,0.3);
    \node[right] at (6.5,0.15) {\small Act};
\end{tikzpicture}
\caption{Mapping the VOXL 2 pipeline to the OODA loop}
\label{fig:voxl2-ooda-loop-mapping}
\end{figure}

The VOXL 2 has support for running Google's Lite Runtime (LiteRT) machine
learning models, formely known as TensorFlow Lite (TFLite). Trained TensorFlow
and PyTorch models can be converted to LiteRT models by using techniques such
as quantization and pruning \cite{jacob2017}. Float16 quantization reduces the
floating-point precision of the model's weights to 16-bits, and full integer
quantization converts them to integers. This reduces the model's size, memory
usage and inference latency, making it more suitable for mobile device
inferencing. Float16 quantization, for instance, reduces the size of the model
by half and causes minimal loss in accuracy. However, GPU acceleration of
float16-quantized models requires half-precision floating point support (FP16)
\cite{ho2017}.

\Cref{fig:voxl2-ooda-loop-mapping} depicts the new pipeline with the VOXL 2.
In the existing SteelEagle architecture, as described in
\cref{sec:steeleagle-bg}, video decoding is performed on the cloudlet since all
video stream network packets are forwarded to the cloudlet using the Onion
Omega as a network gateway. To perform onboard computation on the VOXL 2, we
must obtain individual frames from the drone's video stream.  This requires
decoding the H.264 video stream on the VOXL 2. The Orient phase begins after
the frames are decoded.  The VOXL pipeline has an inner and outer Orient loop:
\begin{itemize}
    \item \textbf{Inner loop}: Corresponds to components Orient$_e$ and
        Orient$_f$, which involve analysis of the decoded frame and planning
        on the VOXL
    \item \textbf{Outer loop}: Corresponds to components Orient$_w$ through
        Orient$_z$. Involves offloaded analysis of the decoded frame and
        planning on the cloudlet
\end{itemize}

The outer loop involves offloading to the cloudlet, which incurs latency and
transmission delay to and from the cloudlet. The cloudlet, being a stationary
device, can run more accurate analysis and planning phases in less time.

To conserve bandwidth, the outer loop is not always exercised.  The pipeline
can leverage offload shaping techniques (\cref{sec:offload-shaping}) to
determine if the frame should be discarded. Then, a remote execution system,
such as the one developed by Flinn et al \cite{flinn2002}, can determine if
processing should be offloaded. Once frame processing is complete, the pipeline
determines whether any drone actuation needs to be performed, and generates the
corresponding command.

\subsection{Onboard Computation Design}

\begin{figure}[htbp]
\centering
\includegraphics[width = .9\textwidth]{figs/onboard-design-crop.pdf}
\begin{captext}
Components in red are developed in Python, and components in green in C++
\end{captext}
\caption{VOXL 2 Onboard Computation Pipeline}
\label{fig:voxl2-onboard-pipeline}
\end{figure}

The VOXL 2 supports the Open Computing Language (OpenCL) standard, which
assists in writing parallel programs which leverage its hardware accelerators,
including its GPU. The VOXL 2 ships with a software suite that includes various
programs designed to be run as \texttt{systemd} services. One of these
programs, VOXL TFlite Server, performs hardware-accelerated machine learning
inference. The inputs and outputs of the programs are sent through an
inter-process communication library called \texttt{libmodalpipe}, which uses
Unix pipes. \texttt{libmodalpipe} includes data structures that represent data
items such as camera image metadata and the results of inference. For example,
\texttt{camera\_image\_metadata\_t} specifies image metadata such as width,
height, size and format of the image, and \texttt{ai\_detection\_t} includes
information relevant to object detection models, such as detected classes,
confidence levels, and bounding boxes.

VOXL TFlite Server is designed to take the drone's camera data continuously as
input through a Unix pipe and output detections through another Unix pipe. Any
service interesting in consuming the inference results can register as a client
to the output pipe. We can utilize this framework to perform onboard inference,
publishing frames requiring inference to a pipe configured as input to VOXL
TFLite Server.

The VOXL software ecosystem is targeted for C/C++ programs. However, the
SteelEagle pipeline is currently developed in Python. As a result, a C++ proxy
is required that can receive frames from the SteelEagle program and forward
them to the VOXL TFLite Server for inference.  As
\cref{fig:voxl2-onboard-pipeline} shows, an onboard compute module was
developed that communicates with SteelEagle using the ZeroMQ messaging library,
listening for inference requests. ZeroMQ has bindings available for popular
programming languages, making it suitable for this use case. A SteelEagle stub
provides a convenience API that generates a \texttt{protobuf} message that is
sent to the onboard compute module. The \texttt{protobuf} message includes the
contents of the frame as an RGB image along with frame metadata such as width
and height. The onboard compute client generates the
\texttt{camera\_image\_metadata\_t} structure that VOXL TFLite Server expects,
and sends it to its input pipe along with the frame data.

Once the VOXL TFLite Server receives an input, it publishes one
\texttt{ai\_detection\_t} data structure to the output pipe for each detection
on a given frame. There is no metadata published that specifies the number of
detections to expect. This works well if a consumer wants a live stream of
object detections. However, in our case, we only want to reply to the
SteelEagle program once we receive all detections for a given frame. To achieve
this, the VOXL TFLite Server program was modified to send a delimiter
\texttt{ai\_detection\_t} data structure once all the detections have been sent
for a given frame. Once this delimiter is received, the onboard compute module
adds all the detections to a single \texttt{protobuf} message, and sends it
back to the SteelEagle program.

\subsection{OODA Loop of VOXL 2 SteelEagle Pipeline}

Using the terminology from \cref{fig:voxl2-ooda-loop-mapping}, we consider the
performance of each component of the VOXL pipeline.

\subsubsection*{Observe$_{ab}$}

The Observe$_{ab}$ component of the OODA loop in the new pipeline is the same as in
the case of the Onion Omega (\cref{sec:onion-observe-ab}), including on-drone
sensing and pre-processing, which involves generation of an RTSP H.264 video
stream from the drone camera's raw video frames. We use the same drone,
the Parrot Anafi, for the VOXL pipeline. As before, this component has a mean
of 253 ms, with a standard deviation of 12 ms, and a p99 of 277 ms.

\subsubsection*{Observe$_{c}$}

Observe$_c$ involves a short Wi-Fi segment from the drone to the VOXL 2. Since
the VOXL 2 is mere centimeters away from the drone, this component is
negligible.

\begin{comment}
\Cref{fig:voxl2-decoding-performance} shows the latency measurements. The
latency distribution has a mean of 262 ms, with a standard deviation of 10.5 ms
and a p99 of 282 ms. This is comparable to the Observe$_{ab}$ measurements from
the Onion Omega pipeline.
\end{comment}

\subsubsection*{Orient+Decide$_{d}$}

\begin{figure}[htbp]
    \centering
    \includegraphics[width = .4\textwidth]{figs/voxl-decoding-time.pdf}\\
\small{Mean: 54.84$\pm$11.59~ms\; p99: 76.47~ms}\\
\caption{Orient+Decide$_d$ Measurements}
\label{fig:voxl2-decoding-latency}
\end{figure}

Orient+Decide$_d$ involves the decoding of the RTSP video stream generated
by the drone. \Cref{fig:voxl2-decoding-latency} shows the measurements of
the video decoding. The latency distribution has a mean of 55 ms, with a
standard deviation of 11.6 ms, and a p99 of 76.5 ms.

\subsubsection*{Orient+Decide$_{e}$}

\begin{figure}[htbp]
    \centering
    \includegraphics[width = .4\textwidth]{figs/voxl-5g-latency.pdf}\\
\small{Mean: 37.52$\pm$10.09~ms\; p99: 59.24~ms}\\
\caption{Orient+Decide$_e$ Measurements}
\label{fig:voxl2-5g-latency}
\end{figure}

Orient+Decide$_e$ is incurred when the remote execution system decides to
offload computation, and involves a 5G segment from the VOXL 2 to the cloudlet.

\subsubsection*{Orient+Decide$_{f}$}

\begin{figure}[htbp]
\centerline{\includegraphics[width = 0.5\textwidth]{figs/onboard-inference-hist.pdf}}
\centering
Mean: 60.5$\pm$6.77~ms\; p99: 77.5~ms\\
\caption{Onboard inference on the VOXL 2 using a Float16-quantized YOLOv5 model}
\label{fig:voxl2-inference-hist}
\end{figure}

Orient+Decide$_f$ corresponds to Stage 2 and Stage 3 of the Onion Omega pipeline, discussed in \cref{sec:onion-orient-decide-d}. \Cref{fig:voxl2-inference-hist} shows measurements for running a machine learning model on the VOXL. We use a float16-quantized YOLOv5 model for inference, which is used for detecting objects. We obtain a mean latency of 60.5 ms, with a standard deviation of 6.77 ms and a p99 of 77.5 ms.

\subsubsection*{Act$_{g}$}

Act$_g$ involves a 5G segment from the cloudlet to the drone via the VOXL 2 when offloading to the cloudlet. In the case of the onboard pipeline, only a Wi-Fi segment is needed.

\subsubsection*{Act$_{hi}$}

Act$_{hi}$ is the same as Act$_{fg}$ in the Onion Omega pipeline, discussed in \cref{sec:onion-act-fg}, involving drone post-processing and actuation. For the actuation of the Parrot Anafi drone's gimbal, it has a mean of 173 ms and a standard deviation of 15 ms.

\subsection{Latency of VOXL 2 Offload Pipeline}

In the VOXL pipeline, frames are decoded on the VOXL instead of the cloudlet.
This means that if the remote execution system decides to offload computation
to the cloudlet, it must send individual frames to the cloudlet. This increases
bandwidth requirements substantially. Previously, a bandwidth of 5 Mbps was
sufficient to send the video stream to the cloudlet. Individual 720p frames
average about 180 KB when encoded as JPEG, requiring a bandwidth of 43.2 Mbps
to send 30 frames per second.  As \cref{tab:voxl2-offload-performance} shows,
the VOXL 2 does not have sufficient bandwidth to sustain this frame rate.
Reducing frame size allows us to achieve this frame rate.

\begin{table}[htbp]
    \centering
    \begin{tabular}{llll}
        \toprule
        \textbf{Frame details} & \textbf{Frame size (KB)} & \textbf{Avg. Latency (ms)} & \textbf{Achievable FPS}\\
        \midrule
        720p & 180 & 934$\pm$\small{30.47} & 12.58\\
        720p, 80\% JPG quality  & 98 & 600$\pm$\small{65.76} &28.63\\
        720p, 70\% JPG quality & 79 & 304$\pm$\small{12.82} & 29.95\\
        360p & 79 & 307$\pm$\small{20.60}  &29.89 \\
        360p, 80\% JPG quality & 38 &288$\pm$\small{13.96} & 29.89\\
        360p, 70\% JPG quality & 31 & 286$\pm$\small{17.26} & 29.85\\
        \bottomrule
\end{tabular}
\caption{Offloading Performance on the VOXL 2}
\label{tab:voxl2-offload-performance}
\end{table}

\chapter{Conclusion and Future Work}
\label{ch:conclusion}

We end with concluding remarks and a discussion of future work.

\section{Conclusion}

We performed a detailed investigation into the SteelEagle pipeline,
measuring the latency and bandwidth of various components. We identified
the FFmpeg video decoding software as a system bottleneck, and switched to
a different software that reduced drone-to-cloudlet latency by a factor of two.
Then, we performed a more systematic benchmark of the pipeline, mapping its
components to the OODA loop framework. We found the onboard decoding of the
drone video stream to be the next bottleneck in the system.

Currently, this onboard decoding makes up almost half of the total OODA loop
latency. The agility of the system will be handicapped by this component until
we new drones are designed with an emphasis on minimizing this latency. The
use of more powerful onboard ASIC could help. The use of advanced sensor
technologies that are optimized for speedy sensor readout could also be
beneficial.

We also explored integrating the VOXL system-on-a-chip device as a payload
on SteelEagle drones. This requires us to deal with bandwidth challenges,
as sending full-fidelity frames over the network requires a very high
bandwidth.

\section{Future Work}

There are many avenues for future work.
\begin{itemize}
    \item Exploring new drones for use in the SteelEagle system that have a
        lower onboard sensing and pre-processing latency
    \item Implementing a remote execution system that takes into account
        current conditions to decide whether to offload computation
\end{itemize}


%\appendix
%\include{appendix}

\backmatter
%\renewcommand{\baselinestretch}{1.0}\normalsize

% By default \bibsection is \chapter*, but we really want this to show
% up in the table of contents and pdf bookmarks.
\renewcommand{\bibsection}{\chapter{\bibname}}
%\newcommand{\bibpreamble}{This text goes between the ``Bibliography''
%  header and the actual list of references}
\bibliographystyle{unsrt}
\bibliography{register} %your bib file

% \begin{thebibliography}{00}
%
% \bibitem{thing} thing
%
% \end{thebibliography}

\end{document}
