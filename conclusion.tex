\chapter{Conclusion and Future Work}
\label{ch:conclusion}

We end with concluding remarks and a discussion of future work.

\section{Conclusion}

We performed a detailed investigation into the SteelEagle pipeline,
measuring the latency and bandwidth of various components. We identified
the FFmpeg video decoding software as a system bottleneck, and switched to
a different software that reduced drone-to-cloudlet latency by a factor of two.
Then, we performed a more systematic benchmark of the pipeline, mapping its
components to the OODA loop framework. We found the onboard decoding of the
drone video stream to be the next bottleneck in the system.

Currently, this onboard decoding makes up almost half of the total OODA loop
latency. The agility of the system will be handicapped by this component until
we new drones are designed with an emphasis on minimizing this latency. The
use of more powerful onboard ASIC could help. The use of advanced sensor
technologies that are optimized for speedy sensor readout could also be
beneficial.

We also explored integrating the VOXL system-on-a-chip device as a payload
on SteelEagle drones. This requires us to deal with bandwidth challenges,
as sending full-fidelity frames over the network requires a very high
bandwidth.

\section{Future Work}

There are many avenues for future work.
\begin{itemize}
    \item Exploring new drones for use in the SteelEagle system that have a
        lower onboard sensing and pre-processing latency
    \item Implementing a remote execution system that takes into account
        current conditions to decide whether to offload computation
\end{itemize}
