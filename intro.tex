\chapter{Introduction}
% Goal : 5-10 page talk before the talk
% maybe not 10 pages, but providing a general overview of the paper is the goal here
A significant body of work has aimed to address the inherent resource poverty
of mobile devices using cloud or cloudlet offload \cite{satya}. Mobile
devices perpetually lag behind static devices in computational ability because
of their size and weight constraints \cite{satya2014}. Increased computational
ability, at the same level of hardware efficiency, demands higher energy
consumption. This requires larger batteries to preserve operating time, thereby
increasing weight and size.  However, mobile devices become less useful the
heavier and bigger they become.  For example, people prefer lighter and thinner
smartphones as they are less intrusive and more portable. It turns out to also
true for unmanned aerial vehicles (UAVs), or drones, which spend most of their
energy on flight.  Increased drone weight leads to an upwards spiral in weight
as larger rotors and more powerful motors are needed to achieve the same amount
of lift.  This, in turn, requires larger batteries, which in turn may require a
more reinforced aircraft structure, and so on.

Cloudlet offload addresses these issues by allowing mobile devices to remotely
execute computationally expensive tasks on static infrastructure that does not
need to be light or small. This allows mobile devices to retain their low
weight and small size, but possess computational abilities way beyond what
would otherwise be possible.

SteelEagle, an autonomous drone system, follows this approach \cite{bala2024}.
Its main tenet is the use of commerical-off-the-shelf (COTS) drones, which are
readily available at low cost but typically have minimal to no on-board
compute.  Utilizing cloudlet offload, SteelEagle enables COTS drones to perform
much more intelligent tasks than what they were designed for and exhibit
autonomous capabilities typically only found in larger, more expensive drones.
SteelEagle focuses on tasks that require the ability to sense changes in the
physical environment over time, and to react to them. During flight, for
instance, a drone can relay the video stream from its camera over a commercial
cellular network to an on-ground cloudlet over a commercial cellular network,
which can run a complex pipeline involving neural networks to perform tasks
such as object detection.  This processing of the raw drone sensor streams
allows us to obtain higher level semantic information about the drone's
physical environment, such as the distance to obstacles that present the danger
of a collision. The drone can then be sent piloting commands to react. For
instance, these could be instructions to actuate to avoid an impending
collision.

An important consideration for SteelEagle is the agility of the resulting
system.  How quickly can a drone respond to a change in its environment? It is
the end-to-end latency of the entire execution pipeline, including sensing,
offloading, inferencing, decision making, and actuation, that defines the
agility. A high end-to-end latency can severely handicap the drone---it
must fly at lower speeds to be safe if it takes a long time to identify
obstacles and actuate to bypass them.

To better understand the end-to-end latency, we can utilize the "Observe,
Orient, Decide, Act" (OODA) loop framework devised by military strategist John
R. Boyd. According to Boyd, decision-making happens in a continous iteration of
these steps. He attributed the faster OODA loop of U.S. pilots flyng F-86s,
because of the bubble-shaped canopy offering better visibility and hydraulic
controls that allowed for easier switching between manoeuvres, as the reason
why the slower F-86s fared better than the North Korean MiG-15s during the
Korean War \cite{morton1995}. Paritioning the SteelEagle execution pipeline
according to the OODA steps


Cloudlet offload, by itself, imposes an
upper bound on drone agility as it adds the cost of round-trip latency to a
cloudlet. In the case of SteelEagle, this cost has so far been unavoidable
as there has been no compute available on the drone.

\section{Motivation}

However, using only cloudlet offload presents severe limitations. First,
SteelEagle drones struggle in areas with unreliable cell service and are
completely inoperable in regions without cellular coverage. Second, cloudlet
offload imposes an upper bound on drone agility as it adds the cost of a
round-trip latency to a cloudlet. As originally envisioned, cloudlets are
differentiated from clouds because of their physical proximity, which allows
application end-to-end response time to be just a few milliseconds. However,
in practice, the usage of commerical cellular networks for offload to the
cloudlet increases this latency to tens of milliseconds. In practice, this
limits the agility of the drone, as the reaction time is at least the round-trip
time to the cloudlet.

\section{Overview}


\section{Contributions}

The contributions of this thesis include
\begin{itemize}
  \item Things
\end{itemize}


\section{Organization}
